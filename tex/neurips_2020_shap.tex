\documentclass{article}

% if you need to pass options to natbib, use, e.g.:
%     \PassOptionsToPackage{numbers, compress}{natbib}
% before loading neurips_2020

% ready for submission
% \usepackage{neurips_2020}

% to compile a preprint version, e.g., for submission to arXiv, add add the
% [preprint] option:
%     \usepackage[preprint]{neurips_2020}

% to compile a camera-ready version, add the [final] option, e.g.:
%     \usepackage[final]{neurips_2020}

% to avoid loading the natbib package, add option nonatbib:

\usepackage[nonatbib]{neurips_2020}

\usepackage[utf8]{inputenc} % allow utf-8 input
\usepackage[T1]{fontenc}    % use 8-bit T1 fonts
\usepackage{hyperref}       % hyperlinks
\usepackage{url}            % simple URL typesetting
\usepackage{booktabs}       % professional-quality tables
\usepackage{amsfonts}       % blackboard math symbols
\usepackage{nicefrac}       % compact symbols for 1/2, etc.
\usepackage{microtype}      % microtypography

\usepackage{amsfonts}
\usepackage{bm}
\usepackage{tikz}
\usetikzlibrary{bayesnet}
\usetikzlibrary{arrows}
\usepackage{color}
\usepackage{graphicx}
\usepackage{caption}
\usepackage{subcaption}
\usetikzlibrary{backgrounds}
\usepackage{multirow}
\usepackage{algorithm}
\usepackage{algpseudocode}
\usepackage{amsthm}
\usepackage{mathtools}

\newcommand{\een}{\mathbb{1}}
\newcommand{\vX}{\mathbf{X}}
\newcommand{\vx}{\mathbf{x}}
\newcommand{\vq}{\mathbf{q}}
\newcommand{\vf}{\mathbf{f}}
\newcommand{\ve}{\bm{\epsilon}}
\newcommand{\expectation}{\mathbb{E}}
\newcommand{\contribution}{{\phi}}
\newcommand{\val}{{v}}
\newcommand{\dodo}{\mathit{do}}
\newcommand{\ldo}[1]{\dodo(X_{#1} = x_{#1})}
\newcommand{\lvdo}[1]{\dodo(\vX_{#1} = \vx_{#1})}
\newcommand{\sdo}[1]{\hat{x}_{#1}}
\newcommand{\svdo}[1]{\hat{\vx}_{#1}}
\newcommand{\pa}{\mathop{\textit{pa}}}
\newcommand{\spa}{\mathop{\textit{\scriptsize pa}}}
\newcommand{\perm}{\pi}
\newcommand{\actcont}{\contribution^{\mbox{\scriptsize active}}}
\newcommand{\passcont}{\contribution^{\mbox{\scriptsize passive}}}
\newcommand{\operator}{\mathit{op}}
\newcommand{\sop}[1]{\operator(x_{#1})}
\newcommand{\svop}[1]{\operator(\vx_{#1})}
\newcommand{\lop}[1]{\operator(X_{#1} = x_{#1})}
\newcommand{\lvop}[1]{\operator(\vX_{#1} = \vx_{#1})}
\newcommand{\allfeatures}{{N}}
\newcommand{\bx}{\bar{x}}
\newcommand{\tx}{\tilde{x}}
\newcommand{\hx}{\hat{x}}
\newcommand{\allmeans}{{\cal X}}
\newcommand{\diagbeta}{{\cal B}}
\newcommand{\mapmat}{{\cal M}}
\newcommand{\contmat}{{\cal C}}
\newcommand{\onder}[2]{{#1}_{\mbox{\scriptsize #2}}}
\newcommand{\boven}[2]{#1^{\mbox{\scriptsize #2}}}
\newcommand{\isequal}{\hspace*{-2.5mm} & = & \hspace*{-2.5mm}}
\newcommand{\chaincomponents}{{\cal T}}
\newcommand{\isequaldo}[1]{\hspace*{-2.5mm} & \overset{(#1)}{=} & \hspace*{-2.5mm}}
\newcommand{\Spre}{\underline{S}}
\newcommand{\Spost}{\bar{S}}

\newcommand{\comment}[1]{{\color{red} #1}}


\newtheorem{theorem}{Theorem}
\newtheorem{corollary}[theorem]{Corollary}

\title{Causal Shapley values}

% The \author macro works with any number of authors. There are two commands
% used to separate the names and addresses of multiple authors: \And and \AND.
%
% Using \And between authors leaves it to LaTeX to determine where to break the
% lines. Using \AND forces a line break at that point. So, if LaTeX puts 3 of 4
% authors names on the first line, and the last on the second line, try using
% \AND instead of \And before the third author name.

\author{%
  Pietje Puk\\
  Radboud University
  Institute for Computing and Information Sciences\\
  Nijmegen, The Netherlands \\
  \texttt{pietje.puk@ru.nl} \\
  % examples of more authors
  % \And
  % Coauthor \\
  % Affiliation \\
  % Address \\
  % \texttt{email} \\
  % \AND
  % Coauthor \\
  % Affiliation \\
  % Address \\
  % \texttt{email} \\
  % \And
  % Coauthor \\
  % Affiliation \\
  % Address \\
  % \texttt{email} \\
  % \And
  % Coauthor \\
  % Affiliation \\
  % Address \\
  % \texttt{email} \\
}

\begin{document}

\maketitle

\begin{abstract}
	Shapley values have become one of the most popular model-agnostic methods within explainable artificial intelligence. They explain the output of a possibly complex and highly non-linear machine learning model by attributing the difference between the model output and an average, baseline output to the different features that are used as input to the model. Being based on game-theoretic principles, Shapley values uniquely satisfy several desirable properties. Shapley values are typically computed under the assumption that features are independent, which ignores any causal structure between the features and can lead to unreliable explanations.
	
	In this paper, we propose a novel framework for causal Shapley values that encompasses and sheds new light on recent previous work that aims to relax \comment{or defend???} the independence assumption. Our framework is based on an active, causal interpretation of feature attribution. Replacing conventional conditioning by observation with conditioning by intervention, we call upon Pearl's \textit{do}-calculus to compute the Shapley values, without giving in on any of their desirable properties. We provide a practical implementation for deriving Shapley values based on causal chain graphs and illustrate causal Shapley values on several real-world examples.	\comment{Not completely happy yet. Suggestions welcome!}
\end{abstract}



%Papers may only be up to eight pages long, including figures. Additional pages \emph{containing only a section on the broader impact, acknowledgments and/or cited references} are allowed. Papers that exceed eight pages of content will not be reviewed, or in any other way considered for presentation at the conference.


%The \LaTeX{} style file contains three optional arguments: \verb+final+, which creates a camera-ready copy, \verb+preprint+, which creates a preprint for submission to, e.g., arXiv, and \verb+nonatbib+, which will not load the \verb+natbib+ package for you in case of package clash.

%If you wish to post a preprint of your work online, e.g., on arXiv, using the NeurIPS style, please use the \verb+preprint+ option. This will create a nonanonymized version of your work with the text ``Preprint. Work in progress.'' in the footer. This version may be distributed as you see fit. Please \textbf{do not} use the \verb+final+ option, which should \textbf{only} be used for papers accepted to NeurIPS.

%At submission time, please omit the \verb+final+ and \verb+preprint+options. This will anonymize your submission and add line numbers to aid review. Please do \emph{not} refer to these line numbers in your paper as they will be removed during generation of camera-ready copies.


%There is also a \verb+\paragraph+ command available, which sets the heading in bold, flush left, and inline with the text, with the heading followed by 1\,em of space.

%The \verb+natbib+ package will be loaded for you by default.  Citations may be author/year or numeric, as long as you maintain internal consistency.  As to the format of the references themselves, any style is acceptable as long as it is used consistently.

%Of note is the command \verb+\citet+, which produces citations appropriate for use in inline text.  For example,
%\begin{verbatim}
%   \citet{hasselmo} investigated\dots
%\end{verbatim}
%produces
%\begin{quote}
%  Hasselmo, et al.\ (1995) investigated\dots
%\end{quote}

%If you wish to load the \verb+natbib+ package with options, you may add the following before loading the \verb+neurips_2020+ package:
%\begin{verbatim}
%   \PassOptionsToPackage{options}{natbib}
%\end{verbatim}

%If \verb+natbib+ clashes with another package you load, you can add the optional
%argument \verb+nonatbib+ when loading the style file:
%\begin{verbatim}
%   \usepackage[nonatbib]{neurips_2020}
%\end{verbatim}

%Footnotes should be used sparingly.  If you do require a footnote, indicate footnotes with a number\footnote{Sample of the first footnote.} in the text. Place the footnotes at the bottom of the page on which they appear. Precede the footnote with a horizontal rule of 2~inches (12~picas).
%
%Note that footnotes are properly typeset \emph{after} punctuation
%marks.\footnote{As in this example.}


%Note that publication-quality tables \emph{do not contain vertical rules.} We strongly suggest the use of the \verb+booktabs+ package, which allows for typesetting high-quality, professional tables:
%\begin{center}
%  \url{https://www.ctan.org/pkg/booktabs}
%\end{center}
%This package was used to typeset Table~\ref{sample-table}.
%
%\begin{table}
%  \caption{Sample table title}
%  \label{sample-table}
%  \centering
%  \begin{tabular}{lll}
%    \toprule
%    \multicolumn{2}{c}{Part}                   \\
%    \cmidrule(r){1-2}
%    Name     & Description     & Size ($\mu$m) \\
%    \midrule
%    Dendrite & Input terminal  & $\sim$100     \\
%    Axon     & Output terminal & $\sim$10      \\
%    Soma     & Cell body       & up to $10^6$  \\
%    \bottomrule
%  \end{tabular}
%\end{table}

%Please prepare submission files with paper size ``US Letter,'' and not, for example, ``A4.''

%\begin{itemize}
%
%\item You should directly generate PDF files using \verb+pdflatex+.
%
%\item You can check which fonts a PDF files uses.  In Acrobat Reader, select the
%  menu Files$>$Document Properties$>$Fonts and select Show All Fonts. You can
%  also use the program \verb+pdffonts+ which comes with \verb+xpdf+ and is
%  available out-of-the-box on most Linux machines.
%
%\item The IEEE has recommendations for generating PDF files whose fonts are also
%  acceptable for NeurIPS. Please see
%  \url{http://www.emfield.org/icuwb2010/downloads/IEEE-PDF-SpecV32.pdf}
%
%\item \verb+xfig+ "patterned" shapes are implemented with bitmap fonts.  Use
%  "solid" shapes instead.
%
%\item The \verb+\bbold+ package almost always uses bitmap fonts.  You should use the equivalent AMS Fonts:
%\begin{verbatim}
%   \usepackage{amsfonts}
%\end{verbatim}
%followed by, e.g., \verb+\mathbb{R}+, \verb+\mathbb{N}+, or \verb+\mathbb{C}+
%for $\mathbb{R}$, $\mathbb{N}$ or $\mathbb{C}$.  You can also use the following
%workaround for reals, natural and complex:
%\begin{verbatim}
%   \newcommand{\RR}{I\!\!R} %real numbers
%   \newcommand{\Nat}{I\!\!N} %natural numbers
%   \newcommand{\CC}{I\!\!\!\!C} %complex numbers
%\end{verbatim}
%Note that \verb+amsfonts+ is automatically loaded by the \verb+amssymb+ package.
%
%\end{itemize}

%Most of the margin problems come from figures positioned by hand using \verb+\special+ or other commands. We suggest using the command
%\verb+\includegraphics+ from the \verb+graphicx+ package. Always specify the figure width as a multiple of the line width as in the example below:
%\begin{verbatim}
%   \usepackage[pdftex]{graphicx} ...
%   \includegraphics[width=0.8\linewidth]{myfile.pdf}
%\end{verbatim}
%See Section 4.4 in the graphics bundle documentation
%(\url{http://mirrors.ctan.org/macros/latex/required/graphics/grfguide.pdf})
%
%A number of width problems arise when \LaTeX{} cannot properly hyphenate a
%line. Please give LaTeX hyphenation hints using the \verb+\-+ command when
%necessary.





%Do {\bf not} include this section in the anonymized submission, only in the final paper. You can use the \texttt{ack} environment provided in the style file to autmoatically hide this section in the anonymized submission.
%\end{ack}

%\section*{References}
%
%References follow the acknowledgments. Use unnumbered first-level heading for the references. Any choice of citation style is acceptable as long as you are consistent. It is permissible to reduce the font size to \verb+small+ (9 point)
%when listing the references.
%{\bf Note that the Reference section does not count towards the eight pages of content that are allowed.}
%\medskip

%\small

\section{Introduction}

Explainability. Attribution. Shapley. Conditioning or no conditioning.

\section{A causal interpretation of Shapley values}

\comment{Need more smooth talk.}

We assume that we are given a set of features $\vx$ with corresponding model output $f(\vx)$. We compare this output with the average output
\[
f_0 = \expectation f(\vX) = \int d\vX \: P(\vX) f(\vX) \: ,
\]
with expectation taken over some (for now assumed to be known) probability distribution $P(\vX)$. We would like to attribute the difference between $f(\vx)$ and $f_0$ in a sensible way to the different features $i \in \allfeatures$ with $\allfeatures = \{1,\ldots,n\}$ and $n$ the number of features. That is, we would like to write
\begin{equation}
f(\vx) = f_0 + \sum_{i=1}^n \contribution_i \: ,
\label{eq:efficiency}
\end{equation}
where we will refer to $\contribution_i$ as the contribution of feature $i$ to the output $f(\vx)$. Equation~(\ref{eq:efficiency}) is referred to as the efficiency property~\cite{shapley1953value}, which appears to be a sensible desideratum for any attribution method and we therefore take here as our starting point.

We can think of (at least) two different interpretations on how to go from knowing none of the feature values in $f_0$ to knowing all feature values in $f(\vx)$.
\begin{description}
	\item[Passive.] We interpret the feature vector $\vx$ as a passive observation. Feature values come in one after the other and the contribution of feature $i$ should reflect the difference in expected value of $f(\vX)$ after and before {\em observing} its feature value $x_i$.
	\item[Active.] We interpret the feature vector $\vx$ as the result of an action. Feature values are imposed one after the other and the contribution of feature $i$ relates to the difference in expected value of $f(\vX)$ after and before {\em setting} its value to $x_i$.
\end{description}
Following the above reasoning, the contribution of each feature  depends on the order $\perm$ in which the feature values arrive or are imposed. We write the contribution of feature $i$ given the permutation $\perm$ as
\begin{equation}
\contribution_i(\perm) = \val(\{j: j \preceq_\perm i\}) - \val(\{j: j \prec_\perm i\}) \: ,
\label{eq:contperm}
\end{equation}
with $j \prec_\perm i$ if $j$ precedes $i$ in the permutation $\perm$ and where we define the value function
\begin{equation}
\val(S) = \expectation \left[f(\vX) | \svop{S} \right] = \int d\vX_{\bar{S}} \: P(\vX_{\bar{S}}|\lvop{S}) f(\vX_{\bar{S}},\vx_S) \: .
\label{eq:valuedef}
\end{equation}
Here $S$ is the subset of indices of features with known `in-coalition' feature values $\vx_S$. To compute the expectation, we still need to average over the `out-of-coalition' feature values $\vX_{\bar{S}}$ with $\bar{S} = \allfeatures \setminus S$, the complement of $S$. The operator $\operator()$ specifies how the distribution of the `out-of-coalition' features $\vX_{\bar{S}}$ depends on the `in-coalition' feature values $\vx_{S}$. To arrive at the passive interpretation, we set $\operator()$ to conventional conditioning by observation, yielding $P(\vX_{\bar{S}}|\vx_{S})$. For the active interpretation, we need to condition by intervention, for which we resort to Pearl's \textit{do}-calculus~\cite{pearl1995causal} and write $P(\vX_{\bar{S}}|\lvdo{S})$. \comment{Do we need a longer introduction/explanation of \textit{do}-calculus?} A third option is to ignore the feature values $\vx_S$ and just take the unconditional, marginal distribution $P(\vX_{\bar{S}})$. We will refer to the corresponding Shapley values as conditional, causal, and marginal, respectively.

It is easy to check that the efficiency property~(\ref{eq:efficiency}) holds for any permutation $\perm$.
%\begin{eqnarray*}
%\sum_{i=1}^m \contribution_i(\perm) & = & \sum_{i=1}^m \left(
%\expectation \left[f(\vX) | \svop{\{j:\perm(j) \leq \perm(i)\}} \right] - \expectation \left[f(\vX) | \svop{\{j:\perm(j) < \perm(i)\}} \right] \right) \\
%& = & \sum_{i=1}^m \left(\expectation \left[f(\vX) | \svop{\{j:j \leq i\}} \right] - \expectation \left[f(\vX) | \svop{\{j:j < i\}} \right] \right) \\
%& = & \expectation \left[f(\vX) | \svop{\{j:j \leq m\}} \right] - \expectation \left[f(\vX) | \svop{\{j:j < 1\}} \right] \\
%& = & f(\vx) - \expectation [f(\vX)] \: .
%\end{eqnarray*}
So, for any distribution over permutations $w(\perm)$ with $\sum_{\perm} w(\perm) = 1$, the contributions
\[
\contribution_i = \sum_{\perm} w(\perm) \contribution_i(\perm)
\]
still satisfy~(\ref{eq:efficiency}). An obvious choice would be to take a uniform distribution $w(\perm) = 1/n!$. We then arrive at the standard definition of Shapley values:
\[
\contribution_i = \sum_{S \subseteq \allfeatures\setminus i} \frac{|S|! (n-|S|-1)!}{n!} \left[\val(S \cup i) - \val(S) \right] \: ,
\]
where we use shorthand $i$ for the singleton $\{i\}$. Besides efficiency, these Shapley values uniquely satisfy three other desirable properties~\cite{shapley1953value}.
\begin{description}
	\item[Linearity:] for two value functions $\val_1$ and $\val_2$, we have $\contribution_i(\alpha_1 \val_1 + \alpha_2 \val_2) = \alpha_1 \contribution_i(\val_1) + \alpha_2 \contribution_i(\val_2)$. This guarantees that the Shapley value of a linear ensemble of models is a linear combination of the Shapley values of the individual models.
	\item[Null player (dummy):] if $\val(S \cup i) = \val(S)$ for all $S \subseteq \allfeatures \setminus i$, then $\contribution_i = 0$. A feature that never contributes to the model output receives zero Shapley value.
	\item[Symmetry:] if $\val(S \cup i) = \val(S \cup j)$ for all  $S \subseteq \allfeatures \setminus \{i,j\}$, then $\contribution_i = \contribution_j$. Symmetry holds for marginal, conditional, and causal Shapley values.
\end{description}
Efficiency, linearity, and null player still hold for a non-uniform distribution of permutations, but symmetry is then typically lost. 

\comment{Part of the next two paragraphs probably to introduction.}

Janzing et al.~\cite{janzing2019feature} also argued for an active, interventional interpretation of Shapley values. They make a case for using the marginal instead of the (observational) conditional distribution to compute the Shapley values when dependencies between features are due to confounding. This follows directly from our reasoning, since in models with no causal links between the features and any dependencies only due to confounding, conditioning by intervention reduces to the marginal distribution: $P(\vX_{\bar{S}}|\lvdo{S}) = P(\vX_{\bar{S}})$ for any subset $S$. \comment{Need to discuss~\cite{lundberg2020local}. It claims an intervential Tree SHAP and uses do-calculus notation. Writes ``interventional Tree SHAP (by the laws of causality) enforces an independence between the conditional set S and the set of remaining features $\bar{S}$ ($\vX_S \perp \vX_{\bar{S}}$)''. To be honest, I have no clue where their claim comes from, apart from a reference to Janzing\ldots}

Frye et al.~\cite{frye2019asymmetric} introduce asymmetric Shapley values as a way to incorporate causal information. Instead of taking a uniform distribution over all possible permutations, these asymmetric Shapley values only consider those permutations that are consistent with the causal structure between the features, i.e., are such that a known causal ancestor always precedes its descendants. Frye et al.\ apply conventional conditioning by observation to make sure that the resulting explanations respect the data manifold. This makes the approach somewhat of a mix between an active (incorporating causal structure) and a passive approach (conditioning by observation). In this paper, we suggest a more direct approach to incorporate causality, by replacing conditioning by observation with conditioning by intervention. This can, but does not have to be combined with the idea to make the permutations match the causal structure. We will illustrate the differences in the example below.



\section{Illustration}

\begin{figure}
	\begin{minipage}[t]{0.45\textwidth}
		\centering
		Model A\\[1em]	
		\tikz{
			% nodes
			\node[latent] (y) {$Y$};%
			\node[latent,above=of y,xshift=0cm] (x2) {$X_2$};
			\node[latent,above=of x2,xshift=-1.5cm,yshift=-0.45cm] (x1) {$X_1$};
			\node[latent,above=of y,xshift=1.5cm] (x3) {$X_3$}; 
			\node[latent,above=of x2,xshift=0.75cm,yshift=-0.45cm] (z) {$Z$};
			% edges
			\edge {x1,x2,x3} {y};
			\edge {z,x1} {x2,x3};
			\edge {z} {x3}
			
		}
	\end{minipage}
	\hfill
	\begin{minipage}[t]{0.45\textwidth}
		\centering
		Model B\\[1em]
		\tikz{
			% nodes
			\node[latent] (y) {$Y$};%
			\node[latent,above=of y,xshift=0cm] (x2) {$X_2$};
			\node[latent,above=of x2,xshift=-1.5cm,yshift=-0.45cm] (x1) {$X_1$};
			\node[latent,above=of y,xshift=1.5cm] (x3) {$X_3$};
			\node[obs,above=of x2,xshift=0.75cm,yshift=-0.45cm] (z) {$Z$};
			% edges
			\edge {x1,x2,x3} {y};
			\edge {x1} {x2,x3};
			\edge {x2,x3} {z}
		}
	\end{minipage}
	\caption{Two causal models. In both, $X_1$ causes $X_2$ and $X_3$. In Model A the excess correlation between $X_2$ and $X_3$ is induced by a common confounder $Z$, in Model B by selection bias.}
	\label{fig:linmodel}
\end{figure}

For illustration, we consider two causal models in Figure~\ref{fig:linmodel}. They have a different causal structure, but the same dependency structure (all features are dependent) and we assume that the probability distribution $P(\vX)$ is exactly the same for Model A and Model B. Our estimate of the output $Y$ is a linear function of the features:
\[
f(\vx) = \beta_0 + \sum_{i=1}^3 \beta_i x_i \: .
\]
\comment{Too much detail in this section: move parts to supplement? If so, which parts?}
Combining~(\ref{eq:contperm}) and (\ref{eq:valuedef}), we obtain, after some rewriting
\[
\contribution_i(\perm) =
\beta_i \left(x_i - \expectation [X_i | \svop{j: j \prec_\perm i}]\right) + \sum_{k \succ_\perm i} \beta_k \left( \expectation [X_k | \svop{j: j \preceq_\perm i}] - \expectation [X_k | \svop{j: j \prec_\perm i}] \right) \: .
\]

For marginal Shapley values only the first term before the sum remains, yielding
\[
\contribution_i = \contribution_i(\perm) =
\beta_i (x_i - \expectation [X_i]) \: ,
\]
as also derived in~\cite{aas2019explaining}.

Analytically computing the conditional Shapley values is tedious, but conceptually straightforward. To write the equations in a compact form, we define $\bx_{k|S} = \expectation[X_k|\vx_{S}]$ and combine all expectations in a single matrix $\bar{\allmeans}$:
\[
\bar{\allmeans} = \left(\begin{array}{lll}
x_1 & x_2 & x_3 \\
\bx_1 & \bx_2 & \bx_3 \\
\bx_{1|2} & \bx_{2|1} & \bx_{3|1} \\
\bx_{1|3} & \bx_{2|3} & \bx_{3|2} \\
\bx_{1|2,3} &  \bx_{2|1,3} & \bx_{3|1,2}
\end{array}\right)
%\mbox{~~and~~} \diagbeta =
%\left( \begin{array}{ccc} \beta_1 & 0 & 0 \\ 0 & \beta_2 & 0 \\ 0 & %0 & \beta_3 \end{array} \right)
%\mathop{\textrm{diag}}(\bm{\beta})
\: .
\]
%We similarly define $\hx_{k|S} = \expectation[X_k|\lvdo{S}]$ for expectation under conditioning by intervention.
Any vector $\bm{\contribution}$ with the Shapley values of the three features can be written as a linear combination of these expectations times the regression coefficients $\bm{\beta}$.
With definition of the matrix
\[
\boven{\contmat}{conditional} = {1 \over 6} \left(\begin{array}{rrrrr|rrrrr|rrrrr}
6 & -2 & -1 & -1 & -2  &
0 & -2 & 2 & -1 & 1  &
0 & -2 & 2 & -1 & 1 \\ 
0 & -2 & 2 & -1 & 1  &
6 & -2 & -1 & -1 & -2  &
0 & -2 & -1 & 2 & 1 \\ 
0 & -2 & -1 & 2 & 1  &
0 & -2 & -1 & 2 & 1  &
6 & -2 & -1 & -1 & -2
\end{array}\right) \: ,
\]
the conditional Shapley values for any linear model with three variables and a uniform distribution over permutations can be written as
\begin{equation}
\boven{\bm{\contribution}}{conditional} = \boven{\contmat}{conditional}  \mathop{\textrm{vec}}(\mathop{\textrm{diag}}(\bm{\beta}) \bar{\allmeans}) \: .
\label{eq:shapvec}
\end{equation}
\comment{Skip this explanation?}
%The multiplication with the diagonal matrix of regression coefficients $\bm{\beta} = (\beta_1,\beta_2,\beta_3)$ boils down to multiplying the $i$th column of $\bar{\allmeans}$ with $\beta_i$.
Vectorization stacks the columns on top of one another to end up with a 15-dimensional column vector. The vertical bars in the matrix $\boven{\contmat}{conditional}$ indicate the three blocks, with the first 5 columns in the matrix mapping to the first column of $\bar{\allmeans}$ with expectations of $X_1$, the next 5 columns to the expectations of $X_2$, and the final 5 columns to the expectations of $X_3$.

\comment{Skip this paragraph?}
By summing every column of $\boven{\contmat}{conditional}$, we can perform the sanity check that efficiency indeed holds. The first column of each block (which relates to the feature values themselves) adds up to $1$, the second (corresponding to the marginal expectations) to $-1$, and the other three columns to zero, as they should. Since Shapley values are constructed by always comparing two (possibly) different expectations, each row within each block sums up to zero.

%The independence structure of the models in Figure~\ref{fig:linmodel} simplifies the conditional Shapley values a bit. Since $X_1$ and $X_3$ are independent, we have $\bx_{1|3} = \bx_{1}$ and $\bx_{3|1} = \bx_{3}$. We can still make use of~(\ref{eq:shapvec}) with $\tilde{\allmeans}$ replaced by $\bar{\allmeans}$ changing $\tilde{\contmat}$ into $\bar{\contmat}$ by simply adding the fourth to the second and the thirteenth column to the twelfth column:
%\[
%\bar{\contmat} = {1 \over 6} \left(\begin{array}{rrrrr|rrrrr|rrrrr}
%6 & -3 & -1 & 0 & -2  &
%0 & -2 & 2 & -1 & 1  &
%0 & 0 & 0 & -1 & 1 \\ 
%0 & -3 & 2 & 0 & 1  &
%6 & -2 & -1 & -1 & -2  &
%0 & -3 & 0 & 2 & 1 \\ 
%0 & 0 & -1 & 0 & 1  &
%0 & -2 & -1 & 2 & 1  &
%6 & -3 & 0 & -1 & -2
%\end{array}\right) \: .
%\]

Putting $X_1$ before $X_2$ and $X_3$, and $X_2$ and $X_3$ on equal footing, asymmetric Shapley values only consider the two permutations where $x_1$ is observed before $x_2$ and $x_3$, leading to (we divide by 6 to make it easier to compare with the other Shapley values)
\[
\boven{\contmat}{asymmetric} = {1 \over 6} \left(\begin{array}{rrrrr|rrrrr|rrrrr}
6 & -6 & 0 & 0 & 0  &
0 & -6 & 6 & 0 & 0  &
0 & -6 & 6 & 0 & 0 \\ 
0 & 0 & 0 & 0 & 0  &
6 & 0 & -3 & 0 & -3  &
0 & 0 & -3 & 0 & 3 \\ 
0 & 0 & 0 & 0 & 0  &
0 & 0 & -3 & 0 & 3  &
6 & 0 & -3 & 0 & -3
\end{array}\right) \: .
\]

\begin{table}
	\begin{center}
		\begin{tabular}{r|cc} \toprule
			expectation & model A & model B \\ \midrule
			$\hx_{1|2}$ & \multicolumn{2}{c}{$\bx_1$} \\
			$\hx_{1|3}$ & \multicolumn{2}{c}{$\bx_1$} \\
			$\hx_{1|2,3}$ & \multicolumn{2}{c}{$\bx_1$} \\ \midrule
			$\hx_{2|1}$ & \multicolumn{2}{c}{$\bx_{2|1}$} \\
			$\hx_{2|3}$ & $\bx_{2}$ & $\bx_{2|3}$ \\
			$\hx_{2|1,3}$ & $\bx_{2|1}$ & $\bx_{2|1,3}$ \\ \midrule
			$\hx_{3|1}$ & \multicolumn{2}{c}{$\bx_{3|1}$} \\
			$\hx_{3|2}$ & $\bx_{3}$ & $\bx_{3|2}$ \\
			$\hx_{3|1,2}$ & $\bx_{3|1}$ & $\bx_{3|1,2}$ \\ \bottomrule
		\end{tabular}
	\end{center}
	\caption{Turning expectations under conditioning by intervention, $\hx_{i|S} = \expectation[x_i|\lvdo{S}]$, into expectations under conventional conditioning by observation, $\bx_{i|S} = \expectation[x_i|\vX_S]$, for the two models in Figure~\ref{fig:linmodel}. \comment{Needed? Can put it next to Figure~\ref{fig:linmodel} to save space.}}
	\label{tab:rewriting}
\end{table}

%Since the independence structure for Model A and Model B is the same, we end up with the same conditional and asymmetric Shapley values. Since the causal structure is different, we expect the causal Shapley values for Model A and Model B to be different.
Using the standard rules for \textit{do}-calculus~\cite{pearl1995causal}, we show in Table~\ref{tab:rewriting} how the expectations under conditioning by intervention reduce to expectations under conditioning by observation. Since in Model A the correlation between $X_2$ and $X_3$ is due to confounding, the interventional expectations and thus Shapley values simplify considerably:
\[
\boven{\contmat}{causal,A} = {1 \over 6} \left(\begin{array}{rrrrr|rrrrr|rrrrr}
6 & -6 & 0 & 0 & 0  &
0 & -3 & 3 & 0 & 0  &
0 & -3 & 3 & 0 & 0 \\ 
0 & 0 & 0 & 0 & 0  &
6 & -3 & -3 & 0 & 0  &
0 & 0 & 0 & 0 & 0 \\ 
0 & 0 & 0 & 0 & 0  &
0 & 0 & 0 & 0 & 0  &
6 & -3 & 3 & 0 & 0
\end{array}\right) \: .
\]
For Model B, on the other hand, features $X_2$ and $X_3$ do affect each other when intervened upon, which makes that compared to the conditional Shapley values only the first block changes:
\[
\boven{\contmat}{causal,B} = {1 \over 6} \left(\begin{array}{rrrrr|rrrrr|rrrrr}
6 & -6 & 0 & 0 & 0 &
0 & -2 & 2 & -1 & 1  &
0 & -2 & 2 & -1 & 1 \\ 
0 & 0 & 0 & 0 & 0  &
6 & -2 & -1 & -1 & -2  &
0 & -2 & -1 & 2 & 1 \\ 
0 & 0 & 0 & 0 & 0  &
0 & -2 & -1 & 2 & 1  &
6 & -2 & -1 & -1 & -2
\end{array}\right) \: .
\]

To make this more concrete, let us assume that $P(\vX)$ follows a multivariate normal distribution corresponding to the causal model
\[
X_1 \sim {\cal N}(0;1) \mbox{~~and~~} (X_2,X_3|X_1) \sim {\cal N}\left((\alpha_2 X_1, \alpha_3 X_1); \left(\begin{array}{cc} 1 & \rho \\ \rho & 1 \end{array} \right) \right) \: ,
\]
where for notational convenience, we chose zero means and unit variance for all noise variables. The first feature drives the second and third feature with coefficients $\alpha_2$ and $\alpha_3$. The confounding in Model A or selection bias in Model B  leads to correlation $\rho$ on top of the correlation induced by the joint dependence on the first feature. Straightforward calculations yield
\[
\bar{\allmeans} = \left(\begin{array}{ccc}
x_1 & x_2 & x_3 \\
0 & 0 & 0 \\
{\alpha_2 \over \sigma_2^2} x_2 & \alpha_2 x_1 & \alpha_3 x_1 \\
{\alpha_3 \over \sigma_3^2} x_3 & {\gamma \over \sigma_3^2} x_3 & {\gamma \over \sigma_2^2} x_2 \\
{\delta_{23} x_2 + \delta_{32} x_3 \over 1 - \rho^2 + \delta_{23} \alpha_2 + \delta_{32} \alpha_3} & (\alpha_2 - \rho \alpha_3) x_1 + \rho x_3 & (\alpha_3 - \rho \alpha_2) x_1 + \rho x_2
\end{array}\right)
\: .
\]
with $\delta_{ij} = \alpha_i - \rho \alpha_j$, $\sigma_i^2 = 1 + \alpha_i^2$ (the marginal variance for feature $i$), and $\gamma = \rho + \alpha_2 \alpha_3$ (the total covariance between the second and third feature). Plugging this and the expressions for the different $\contmat$ matrices into~(\ref{eq:shapvec}), we obtain the asymmetric Shapley values
\begin{eqnarray*}
	\boven{\contribution_1}{asymmetric} \isequal \beta_1 x_1 + (\alpha_2 \beta_2 + \alpha_3 \beta_3) x_1  \\
	\boven{\contribution_2}{asymmetric} \isequal \beta_2 x_2 - \alpha_2 \beta_2 x_1 + {\rho \over 2} (\alpha_3 \beta_2 - \alpha_2 \beta_3) x_1 + {\rho \over 2} (\beta_3 x_2 - \beta_2 x_3) \\
	\boven{\contribution_3}{asymmetric} \isequal \beta_3 x_3 - \alpha_3 \beta_3 x_1 + {\rho \over 2} (\alpha_2 \beta_3 - \alpha_3 \beta_2) x_1 + {\rho \over 2}  (\beta_2 x_3 - \beta_3 x_2) \: ,
\end{eqnarray*}
and the causal Shapley values, for Model A,
\begin{eqnarray*}
	\boven{\contribution_1}{causal,A} \isequal \beta_1 x_1 + {1 \over 2} (\alpha_2 \beta_2 + \alpha_3  \beta_3) x_1  \\
	\boven{\contribution_2}{causal,A} \isequal \beta_2 x_2 - {1 \over 2} \alpha_2 \beta_2 x_1 \\
	\boven{\contribution_3}{causal,A} \isequal \beta_3 x_3 - {1 \over 2} \alpha_3 \beta_3 x_1  \: .
\end{eqnarray*}
and, for Model B, \comment{(really ugly\ldots can we prevent this?)}
\begin{eqnarray*}
	\boven{\contribution_1}{causal, B} \isequal \beta_1 x_1 + {1 \over 2} (\alpha_2 \beta_2 + \alpha_3 \beta_3) x_1 - {\rho \over 6} (\alpha_3 \beta_2 + \alpha_2 \beta_3) x_1 - {1 \over 6} \left( {\alpha_3 \delta_{23} \over \sigma_3^2} \beta_2 x_3 + {\alpha_2 \delta_{32} \over \sigma_2^2} \beta_3 x_2 \right) \\
	\boven{\contribution_2}{causal, B} \isequal \beta_2 x_2 - {1 \over 2} \alpha_2 \beta_2 x_1 + {\rho \over 6} (2 \alpha_3 \beta_2 - \alpha_2 \beta_3) x_1 + \\
	&& {1 \over 6} \left(2 {\alpha_2 \delta_{32} \over \sigma_2^2} \beta_3 x_2 - {\alpha_3 \delta_{23} \over \sigma_3^2} \beta_2 x_3 \right) + {\rho \over 2} (\beta_3 x_2 - \beta_2 x_3) \\
	\boven{\contribution_3}{causal, B} \isequal \beta_3 x_3 - {1 \over 2} \alpha_3 \beta_3 x_1 + {\rho \over 6} (2 \alpha_2 \beta_3 - \alpha_3 \beta_2) x_1 + \\
	&& {1 \over 6} \left(2 {\alpha_3 \delta_{23} \over \sigma_3^2} \beta_2 x_3 - {\alpha_2 \delta_{32} \over \sigma_2^2} \beta_3 x_2 \right) + {\rho \over 2} (\beta_2 x_3 - \beta_3 x_2) \: .
\end{eqnarray*}

In this linear model, the asymmetric Shapley value for the first feature adds its indirect causal effects on the output through the second and third feature, $\alpha_2 \beta_2 x_1 + \alpha_3 \beta_3 x_1$, to its direct effect, $\beta_1 x_1$. The causal Shapley values for the first feature are somewhat more conservative: they essentially claim only half of the indirect effects through the other two features. \comment{Move the rest of this paragraph elsewhere? Now overlap with direct/indirect in next section.} This is a direct consequence of taking a uniform distribution over all permutations: for any pair of features $i$ and $j$, the feature value $x_i$ is set before and after $x_j$ for exactly half of the number of permutations. Which distribution over permutations to prefer, a uniform one or one that respects the causal structure, depends on the question the practitioner tries to answer and possibly on the application. For example, when a causal link represents a temporal relationship, it may make no sense to set a feature value before the values of all features preceding it in time have been set. In that case, it would be wise to consider a non-uniform distribution over permutations as in~\cite{frye2019asymmetric}. On the other hand, for causal models without temporal interpretation, e.g., describing presumed causal relationships between personal and biomedical variables related to Alzheimer~\cite{shen2020challenges} or between social and economic characteristics in census data~\cite{chiappa2019path}, deviating from a uniform distribution over permutations (and hence sacrificing the symmetry property) seems unnecessary. With or without uniform distribution over permutations, applying \textit{do}-calculus instead of conditioning by observation is a natural way to incorporate causal information.

The Shapley values for Model A are different from those for Model B, even though the observable probability distribution $P(\vX)$ is exactly the same. Those for Model A simplify a lot, because in this model any excess correlation between $X_2$ and $X_3$ beyond the correlation resulting from the common parent $X_1$ results from a confounder. This correlation vanishes when we intervene on either $X_2$ or $X_3$. The contributions of the second and third feature are therefore just their direct effect, minus half of the indirect effect, which already has been attributed to the first feature. In Model B, on the other hand, for most expectations conditioning by intervention reduces to conditioning by observation on the same variables and does not further simplify to conditioning on less or even no variables as for Model A. Since the causal Shapley values consider all six permutations, in contrast to the asymmetric Shapley values which only consider two of them, expectations such as $\expectation[X_2|x_3]$ now also enter the equation, which considerably complicates the analytical expressions.


\section{Causal chain graphs}

\comment{Added a theorem, corollaries, a figure, and an algorithm. Still need to decide which ones to keep and then adapt the text accordingly by adding the appropriate references and removing repetitions.}

Computing causal Shapley values not only requires knowledge of the probability distribution $P(\vX)$, but also of the underlying causal structure. And even then, there is no guarantee that any causal query is identifiable (see e.g., \cite{pearl2012calculus}). For example, if Model A or B also includes a causal link from $X_2$ to $X_3$, even knowing the probability distribution $P(\vX)$ and the causal structure is insufficient: it is impossible to express, for example, $P(X_3|\ldo{2})$ in terms of $P(\vX)$, essentially because, without knowing the parameters of the causal model, there is no way to tell which part of the observed dependence between $X_2$ and $X_3$ is due to the causal link and which due to the confounding or selection bias.

Furthermore, and perhaps more importantly, requiring a practitioner to specify a complete causal structure, possibly even including some of its parameters, would be detrimental to the method's general applicability. We therefore follow the same line of reasoning as in~\cite{frye2019asymmetric} and assume that a practitioner may be able to specify a causal ordering, but not much more.

In the special case that a complete causal ordering of the features can be given and that all causal relationships are unconfounded, $P(\vX)$ satisfies the Markov properties associated with a directed acyclic graph (DAG) and can be written in the form
\[
P(\vX) = \prod_{j \in \allfeatures} P(X_j|\vX_{\spa(j)}) \: ,
\]
with $\pa(j)$ the parents of node $j$. If no further conditional independences are assumed, the parents of $j$ are all nodes that precede $j$ in the causal ordering. For causal DAGs, we have the interventional formula~\cite{lauritzen2002chain}:
\begin{equation}
P(\vX_{\bar{S}}|\lvdo{S}) = \prod_{j \in \bar{S}} P(X_j|\vX_{\spa(j)  \cap \bar{S}},\vx_{\spa(j) \cap S}) \: ,
\label{eq:interventional}
\end{equation}
with $\pa(j) \cap T$ the parents of $j$ that are also part of subset $T$. The interventional formula can be used to answer any causal query of interest. We will often approximate the expectations needed to compute the Shapley values through sampling, which is particularly straightforward for causal DAGs under conditioning by intervention. Variables are sampled consecutively by following the causal ordering. The probability distribution for a feature then only depends on the values of its parents, which by then is either sampled or fixed. Since the intervention blocks the influence of all descendants, there is no need for an MCMC approach such as Gibbs sampling: the values of all features can be sampled in a single pass through the graph.

%If we only consider permutations that follow the causal ordering, as with asymmetric Shapley values, conditioning by intervention reduces to conditioning by observation.

\begin{figure}
	\centering
	\tikz{
		% nodes
		
		\node (t1) at (-3,2) [align=left]{partial causal ordering:\\[1em]
		$\left\{(1,2),(3,4,5),(6,7)\right\}$};
		\node (t2) at (-0.5,2) {\Huge $\Rightarrow$};	
		\node (t3) at (4.5,2) {\Huge $\Rightarrow$};		
		\node (1) at (0,4) [circle,draw,inner sep=0.5mm] {\scriptsize $X_1$};
		\node (2) at (1,4) [circle,draw,inner sep=0.5mm] {\scriptsize $X_2$};
		\node (3) at (1.9,2.8) [circle,draw,inner sep=0.5mm] {\scriptsize $X_3$};
		\node (4) at (3.1,2.8) [circle,draw,inner sep=0.5mm] {\scriptsize $X_4$};
		\node (5) at (2.5,1.8) [circle,draw,inner sep=0.5mm] {\scriptsize $X_5$};
		\node (6) at (1,0.5) [circle,draw,inner sep=0.5mm] {\scriptsize $X_6$};
		\node (7) at (2,0.5) [circle,draw,inner sep=0.5mm] {\scriptsize $X_7$};
		\node (8) [draw,label=above:$\tau_1$,inner sep=0.5mm,fit=(1) (2)] {};
		\node (9) [draw,label=above:$\tau_2$,inner sep=0.5mm,fit=(3) (4) (5)] {};
		\node (10) [draw,label=above:$~~\tau_3$,inner sep=0.5mm,fit=(6) (7)] {};
		\draw (1) -- (2);
		\draw (3) -- (4) -- (5) -- (3);
		\draw (6) -- (7);
		\edge {8} {9};
		\edge {8,9} {10};
%
		\node (11) at (5,4) [circle,draw,inner sep=0.5mm] {\scriptsize $X_1$};
		\node (12) at (6,4) [circle,draw,inner sep=0.5mm] {\scriptsize $X_2$};
		\node (13) at (6.9,2.8) [circle,draw,inner sep=0.5mm] {\scriptsize $X_3$};
		\node (14) at (8.1,2.8) [circle,draw,inner sep=0.5mm] {\scriptsize $X_4$};
		\node (15) at (7.5,1.8) [circle,draw,inner sep=0.5mm] {\scriptsize $X_5$};
		\node (16) at (6,0.5) [circle,draw,inner sep=0.5mm] {\scriptsize $X_6$};
		\node (17) at (7,0.5) [circle,draw,inner sep=0.5mm] {\scriptsize $X_7$};
		\node (21) at (5.5,4.8) [circle,draw,inner sep=0.7mm]
		{\scriptsize $Z_1$};
		\node (22) at (6.5,-0.3) [fill=gray!25,circle,draw,inner sep=0.7mm]
		{\scriptsize $Z_2$};
		\node (18) [draw,label=above:$\tau_1$,inner sep=0.5mm,fit=(11) (12) (21)] {};
		\node (19) [draw,label=above:$\tau_2$,inner sep=0.5mm,fit=(13) (14) (15)] {};
		\node (20) [draw,label=above:$~~\tau_3$,inner sep=0.5mm,fit=(16) (17) (22)] {};
		\edge {18} {19};
		\edge {18,19} {20};
		\edge {21} {11,12};
		\edge {13,14} {15};
		\edge {14,15} {13};
		\edge {15,13} {14};
		\edge {16,17} {22};
		
	}	
	\caption{From partial ordering to causal chain graph. Features on equal footing are combined into a fully connected chain component. How to handle interventions within each component depends on the generative process that best explains the (surplus) dependencies. In this example, the dependency between $X_1$ and $X_2$ in chain component $\tau_1$ is assumed to be the result of a common confounder. The surplus dependencies in $\tau_2$ and $\tau_3$ are assumed to be caused by mutual feedback and selection bias, respectively. \comment{Attempt to illustrate the main ideas. Could be nice, but probably not enough space?}}
	\label{fig:chaingraph}
\end{figure}

We may not always be willing or able to give a complete ordering between the individual variables, but rather a partial ordering as, for example, in Figure~\ref{fig:linmodel} where we have the partial ordering $(\{1\},\{2,3\})$: the first feature precedes the second and third feature in the causal ordering, with the second and third feature on equal footing, i.e., without specifying whether the second causes the third or vice versa. Here causal chain graphs~\cite{lauritzen2002chain} come to the rescue. A causal chain graph has directed and undirected edges. All features that are treated on an equal footing are linked together with undirected edges and become part of the same chain component. Edges between chain components are directed and represent causal relationships. The probability distribution $P(\vX)$ now factorizes as a ``DAG of chain components'':
\[
P(\vX) = \prod_{\tau \in \chaincomponents} P(\vX_\tau|\vX_{\spa(\tau)}) \: ,
\]
with each $\tau$ corresponding to a chain component, consisting of all features that are treated on an equal footing.

How to compute the effect of an intervention now depends on the interpretation of the generative process leading to the (surplus) dependencies between features within each component. If we assume that these are the consequence of marginalizing out a common confounder, as in Model A in Figure~\ref{fig:linmodel}, intervention on a particular feature will break the dependency with the other features. We will refer to the set of chain components for which this applies as $\onder{\chaincomponents}{confounding}$. Another possible interpretation is that the undirected part corresponds to the equilibrium distribution of a dynamic process resulting from interactions between the variables within a component~\cite{lauritzen2002chain}. In this case, setting the value of a feature does affect the distribution of the variables within the same component. The same applies to the case of selection bias, as in Model B in Figure~\ref{fig:linmodel}.

\begin{theorem}
For causal chain graphs, we have the interventional formula
\begin{eqnarray*}
P(\vX_{\bar{S}}|\lvdo{S}) \isequal \prod_{\tau \in \chaincomponents_{\textrm{\upshape \scriptsize confounding}}} P(\vX_{\tau \cap \bar{S}}|\vX_{\spa(\tau)  \cap \bar{S}},\vx_{\spa(\tau) \cap S}) \times \\
&& \prod_{\tau \in \chaincomponents_{\overline{\textrm{\upshape \scriptsize confounding}}}} P(\vX_{\tau \cap \bar{S}}|\vX_{\spa(\tau) \cap \bar{S}},\vx_{\spa(\tau) \cap S},\vx_{\tau \cap S}) \: .
\end{eqnarray*}
\end{theorem}

\begin{proof}
\begin{eqnarray*}
	P(\vX_{\bar{S}}|\lvdo{S}) \isequaldo{1} \prod_{\tau \in \chaincomponents} P(\vX_{\tau \cap \bar{S}}|\vX_{\spa(\tau)  \cap \bar{S}},\lvdo{S}) \hfill \\
	\isequaldo{3}  \prod_{\tau \in \chaincomponents} P(\vX_{\tau \cap \bar{S}}|\vX_{\spa(\tau)  \cap \bar{S}},\lvdo{\spa(\tau) \cap S},\lvdo{\tau \cap S}) \\
	\isequaldo{2}  \prod_{\tau \in \chaincomponents} P(\vX_{\tau \cap \bar{S}}|\vX_{\spa(\tau)  \cap \bar{S}},\vx_{\spa(\tau) \cap S},\lvdo{\tau \cap S}) \: ,
\end{eqnarray*}
where the number above each equal sign refers to the standard \textit{do}-calculus rule from~\cite{pearl2012calculus} that is applied. For a chain component with dependencies induced by a common confounder, rule (3) applies once more and yields
\[
P(\vX_{\tau \cap \bar{S}}|\vX_{\spa(\tau)  \cap \bar{S}},\vx_{\spa(\tau) \cap S}) \: ,
\]
whereas for a chain component with dependencies induced by selection bias or mutual interactions, rule (2) again applies:
\[
P(\vX_{\tau \cap \bar{S}}|\vX_{\spa(\tau)  \cap \bar{S}},\vx_{ \spa(\tau) \cap S},\vx_{\tau \cap S})) \: .
\]
\comment{Proof probably needs to be extended, which is fine when it goes to the supplement anyway.}
\end{proof}

%We should take
%\[
%P(\vX_{\bar{S}}|\lvdo{S}) = \prod_{\tau \in \chaincomponents} P(\vX_{\tau \cap \bar{S}}|\vX_{\spa(\tau)  \cap \bar{S}},\vx_{\spa(\tau) \cap S}) \: ,
%\]
%i.e., we should only condition on the parent components, not on the intervened variables within the same component. This interpretation is consistent with~\cite{janzing2019feature} and Model A in Figure~\ref{fig:linmodel}.
%
%One possible interpretation is that the undirected part corresponds to the equilibrium distribution of a dynamic process resulting from interactions between the variables within a component~\cite{lauritzen2002chain}. In this case, setting the value of a feature does affect the distribution of the variables within the same component. The same applies to the case of selection bias, as in Model B in Figure~\ref{fig:linmodel}. W
%
%
%
%
% and the interventional formula reads
%\[
%P(\vX_{\bar{S}}|\lvdo{S}) = \prod_{\tau \in \chaincomponents} P(\vX_{\tau \cap \bar{S}}|\vX_{\spa(\tau)  \cap \bar{S}},\vx_{\spa(\tau) \cap S},\vx_{\tau \cap S}) \: .
%\]
%That is, in the observational distribution we need to condition on intervened features that are parents and those that are within the same component. The same interventional formula applies when the dependencies within a component are the result of a selection bias as in Model B in Figure~\ref{fig:linmodel}.
%
%However, if we assume that the undirected part of the probability distribution is the consequence of marginalizing out a common confounder, we should take
%\[
%P(\vX_{\bar{S}}|\lvdo{S}) = \prod_{\tau \in \chaincomponents} P(\vX_{\tau \cap \bar{S}}|\vX_{\spa(\tau)  \cap \bar{S}},\vx_{\spa(\tau) \cap S}) \: ,
%\]
%i.e., we should only condition on the parent components, not on the intervened variables within the same component. This interpretation is consistent with~\cite{janzing2019feature} and Model A in Figure~\ref{fig:linmodel}.

\begin{algorithm}
	\caption{Compute the value function $\val(S)$ under conditioning by intervention. \comment{Include in main text? Seems rather obvious\ldots}
%	User has to specify a partial causal ordering of the features, which is translated to a chain graph with components $\chaincomponents$, and for each (non-singleton) component $\tau$ whether or not surplus dependencies are the result of confounding. Other prerequisites include access to the model $f()$, feature vector $\vx$, (procedure to sample from the) probability distribution $P(\vX)$, and number of samples $\onder{N}{samples}$.
}
	\label{alg:sampling}
	\begin{algorithmic}[1]
		\Function{ValueFunction}{$S$}
		\For{$k \gets 1 \mbox{~to~} \onder{N}{samples}$}
		\ForAll{$j \gets 1 \mbox{~to~} |\chaincomponents|$} \Comment{run over all chain components in their causal order}
			\If{confounding$(\tau_j)$} 
				\ForAll{$i \in \tau_j \cap \bar{S}$}
					\State Sample $\tilde{x}_i^{(k)} \sim P(X_i|\tilde{\vx}_{\spa(\tau_j) \cap \bar{S}}^{(k)},\vx_{\spa(\tau_j) \cap \bar{S}})$ \Comment{can be drawn independently}
				\EndFor
			\Else
				\State Sample $\tilde{\vx}_{\tau_j \cap \bar{S}}^{(k)} \sim P(\vX_{\tau_j \cap \bar{S}}|\tilde{\vx}_{\spa(\tau_j) \cap \bar{S}}^{(k)},\vx_{\spa(\tau_j) \cap \bar{S}},\vx_{\tau_j \cap S})$ \Comment{e.g., Gibbs sampling}
			\EndIf
		\EndFor
		\EndFor
		\State $\val \gets {\displaystyle {1 \over \onder{N}{samples}} \sum_{k=1}^{\onder{N}{samples}} f(\vx_S,\tilde{\vx}_{\bar{S}}^{(k)})}$
		\State \Return $\val$
		\EndFunction
	\end{algorithmic}
\end{algorithm}

Theorem~1 connects to observations made and algorithms proposed in recent papers.
\begin{corollary}
	With all features combined in a single component and all dependencies induced by confounding, as in~\cite{janzing2019feature}, causal Shapley values are equivalent to marginal Shapley values. 
\end{corollary}
\begin{proof}
	We immediately get $P(\vX_{\bar{S}}|\lvdo{S}) = P(\vX_{\bar{S}})$ for all subsets $S$, i.e., as if all features are independent.
\end{proof}
\begin{corollary}
	With all features combined in a single component and all dependencies induced by selection bias or mutual interactions, causal Shapley values are equivalent to conditional Shapley values as proposed in~\cite{aas2019explaining}.
\end{corollary}
\begin{proof}
	Now $P(\vX_{\bar{S}}|\lvdo{S}) = P(\vX_{\bar{S}}|\vx_{S})$ for all subsets $S$, which boils down to conventional conditioning by observation.
\end{proof}
\begin{corollary}
	When we only take into account permutations that match the causal ordering and assume that all dependencies within chain components are induced by selection bias or mutual interactions, the resulting asymmetric causal Shapley values are equivalent to the asymmetric conditional Shapley values as defined in~\cite{frye2019asymmetric}.
\end{corollary}
\begin{proof}
	Following~\cite{frye2019asymmetric}, asymmetric Shapley values only include those permutations $\perm$ for which $i \prec_\perm j$ for all known ancestors $i$ of descendants $j$ in the causal graph. For those permutations, we are guaranteed to have
%	$\chaincomponents \cap S \preceq_{{\cal CG}} \chaincomponents \cap \bar{S}$
	$\tau \prec_{\mathcal{CG}} \tau'$ for all $\tau, \tau' \in \mathcal{T}$ such that $\tau \cap S \neq \emptyset, \tau' \cap \bar{S} \neq \emptyset$. That is, the chain components that contain features from $S$ are never later in the causal ordering of the chain graph ${\cal CG}$ than those that contain features from $\bar{S}$. We then have
	\begin{eqnarray*}
	P(\vX_{\bar{S}}|\vx_S) \isequal \prod_{\tau \in \chaincomponents} P(\vX_{\tau \cap \bar{S}}|\vX_{\spa(\tau)  \cap \bar{S}},\vx_{S}) \\
	\isequal \prod_{\tau \in \chaincomponents} P(\vX_{\tau \cap \bar{S}}|\vX_{\spa(\tau)  \cap \bar{S}},\vx_{\spa(\tau) \cap S},\vx_{\tau \cap S}) = P(\vX_{\bar{S}}|\lvdo{S}) \: ,
	\end{eqnarray*}
	where in the last step we used the fact that $\onder{\chaincomponents}{confounding} = \emptyset$.
\end{proof}

Conditioning by intervention in causal chain graphs boils down to conditioning by observation, where features within components that are later in the causal ordering are excluded from the conditioning set. The asymmetric Shapley values in~\cite{frye2019asymmetric}, since still based on conditioning by observation, need to choose a non-uniform distribution over permutations to achieve the same. Our analysis shows that this restriction to permutations that are consistent with the causal ordering is not needed and can be considered a separate choice, orthogonal to conditioning by observation or by intervention.

%In the case of Corollary~1 and \cite{janzing2019feature}, asymmetric conditional Shapley values do not reduce to marginal Shapley values, but to conditional Shapley values. It is not easy to see how this can be fixed by choosing another distribution on the permutations.

With a causal interpretation of Shapley values, we can decompose our explanation to reflect the contribution of direct and indirect effects. Let us consider the contribution $\contribution_i(\perm)$ of a particular permutation $\perm$ and feature $i$. With $\Spre = \{j: j \prec_\perm i\}$ and $\Spost = \{j: j \succ_\perm i\}$, we can write:
\begin{eqnarray*}
\contribution_i(\perm) \isequal
%\val(\Spre \cup i) - \val(\Spre) \\ \isequal
\expectation[f(\vX_{\Spost},\vx_{\Spre \cup i})|\lvdo{\Spre \cup i}] - \expectation[f(\vX_{\Spost \cup i},\vx_{\Spre})|\lvdo{\Spre}] ~~~~~~\mbox{(total effect)} \\
\isequal \expectation[f(\vX_{\Spost},\vx_{\Spre \cup i})|\lvdo{\Spre}] - \expectation[f(\vX_{\Spost \cup i},\vx_{\Spre})|\lvdo{\Spre}] + ~~~~~~~\mbox{(direct effect)} \\
&& \! \! \! \! \expectation[f(\vX_{\Spost},\vx_{\Spre \cup i})|\lvdo{\Spre \cup i}] - \expectation[f(\vX_{\Spost},\vx_{\Spre \cup i})|\lvdo{\Spre}] ~~~~\mbox{(indirect effect)}
\end{eqnarray*}
The direct effect measures the expected change in model output when the stochastic feature $X_i$ is replaced by its feature value $x_i$, without changing the distribution of the other `out-of-coalition' features. The indirect effect measures the difference in expectation when the distribution of the other `out-of-coalition' features changes due to the additional intervention $\ldo{i}$. For any pair of features $(i,j)$ with $j \succ_{\cal CG} i$, the indirect effect of feature $i$ through feature $j$ is added to the Shapley value for $i$, if and only if $j \succ_\perm i$. This goes at the expense of the direct effect of feature $j$, essentially because when feature $j$ is set to its value, the direct effect is computed relative to the expectation conditioned upon intervention given a set of features, including $x_i$. Asymmetric (causal) Shapley values only incorporate permutations $\perm$ with $j \succ_\perm i$ if $j \succ_{\cal CG} i$. With symmetric causal Shapley values, $j \succ_\perm i$ in half of the permutations $\perm$: in the other half feature $j$ is intervened upon before feature $i$ and there is no indirect effect to be accounted for. This makes the indirect effect in asymmetric Shapley values roughly a factor two times the indirect effect in symmetric Shapley values.


\comment{Turn the above into a Theorem? Bit hard to make concrete.}

So, to be able to compute the expectations in the Shapley equations under an interventional interpretation, we need to specify (1) a partial order and (2) whether any dependencies between features that are treated on an equal footing are most likely the result of mutual interaction or selection, or of a common confounder. Based on this information, any expectation by intervention can be translated to an expectation by observation.

To compute these expectations, we can rely on the various methods that have been proposed to compute conditional Shapley values~\cite{aas2019explaining,frye2019asymmetric}. Following~\cite{aas2019explaining}, we will assume a multivariate Gaussian distribution for $P(\vX)$ that we estimate from the training data. Alternative proposals include assuming a Gaussian copula distribution, estimating from the empirical (conditional) distribution (both from~\cite{aas2019explaining}) and a variational autoencoder~\cite{frye2019asymmetric}.

\section{Experiments}

\comment{From here on just rough text and ideas.}

Show that it works. For now: example on bike rental. Do we predict more bike shares on a warm, but cloudy day in August because of the season or because of the weather? \comment{ADNI as another example? Tried German Credit Data, but hard to see differences between causal and conditioning, mainly because the features, such as gender and age, that can be considered causes of some of the others, hardly affect the prediction. Other suggestions?} \comment{Currently using a relatively straightforward adaptation of the code of~\cite{aas2019explaining}. How to describe this? Do we need to publish the code? Do we need to show results for asymmetric Shapley values as well? If so, need to dig deeper into the code. Also: currently no code for handling discrete variables. Could connect to Ruifei's Gaussian copula's for mixed missing data, if needed?}

\section{Discussion}

Conditional and causal Shapley values attribute the difference between the model output and the expected model output to the individual features assuming that their values are passively {\em observed} and actively {\em set}, respectively. Both interpretations are valid, but only the latter enables us to leverage additional available information regarding the causal structure.

Marginal Shapley values are perfectly fine when dependencies are purely the result of confounding, as argued in~\cite{janzing2019feature}. However, when these dependencies are the result of causal relationships or even due to selection bias or mutual feedback, then expectations under conditioning by intervention do not simplify to marginal expectations.

We proposed a novel algorithm to compute causal Shapley values, based on causal chain graphs. All that a practitioner needs to provide is a partial causal order (as for asymmetric Shapley values) and a way to interpret dependencies between features that are on equal footing. Conditioning by intervention becomes conditioning by observation, but only on ancestors and possibly, depending on the interpretation, on features within the same component. Any existing computational approach for conditional Shapley values can be easily adapted and combined with computationally efficient approaches such as KernelSHAP~\cite{lundberg2017unified} and TreeExplainer~\cite{lundberg2020local}. If anything the expectations simplify, since there is no conditioning by observation on the descendants.

Our analysis may also provide a better understanding of asymmetric Shapley values: for all permutations that are consistent with the causal ordering, conditioning by intervention boils down to conditioning by observation. However, the current definition in~\cite{frye2019asymmetric} implicitly assumes that dependencies between features that are on equal footing is due to mutual feedback or selection bias, not common confounding. Whether or not to only consider permutations that match the causal ordering depends on the application and possibly one's preference. This paper shows that it is unnecessary to forgo symmetry in order to arrive at a causal interpretation of Shapley values. Roughly speaking, asymmetric (causal/conditioning) Shapley values attribute all indirect effects of a causal variable through a mediator on the output to the causal variable \comment{(is this the right term?)}, subtracting them from the Shapley value of the mediator. Symmetric Shapley values are more conservative and attribute only half to the causal variable.

\comment{Discuss non-manipulable causes as in~\cite{pearl2018obesity}?}
	


\comment{Compare with counterfactual explanations?}

%\section*{Broader Impact}
%
%Authors are required to include a statement of the broader impact of their work, including its ethical aspects and future societal consequences. 
%Authors should discuss both positive and negative outcomes, if any. For instance, authors should discuss a) 
%who may benefit from this research, b) who may be put at disadvantage from this research, c) what are the consequences of failure of the system, and d) whether the task/method leverages
%biases in the data. If authors believe this is not applicable to them, authors can simply state this.

%Use unnumbered first level headings for this section, which should go at the end of the paper. {\bf Note that this section does not count towards the eight pages of content that are allowed.}

%\begin{ack}
%Use unnumbered first level headings for the acknowledgments. All acknowledgments
%go at the end of the paper before the list of references. Moreover, you are required to declare 
%funding (financial activities supporting the submitted work) and competing interests (related financial activities outside the submitted work). 
%More information about this disclosure can be found at: \url{https://neurips.cc/Conferences/2020/PaperInformation/FundingDisclosure}.

\bibliography{shapleyrefs}
\bibliographystyle{plain}



\end{document}
